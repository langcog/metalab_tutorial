%
% Annual Cognitive Science Conference
% Sample LaTeX Two-Page Summary -- Proceedings Format
%

% Original : Ashwin Ram (ashwin@cc.gatech.edu)       04/01/1994
% Modified : Johanna Moore (jmoore@cs.pitt.edu)      03/17/1995
% Modified : David Noelle (noelle@ucsd.edu)          03/15/1996
% Modified : Pat Langley (langley@cs.stanford.edu)   01/26/1997
% Latex2e corrections by Ramin Charles Nakisa        01/28/1997
% Modified : Tina Eliassi-Rad (eliassi@cs.wisc.edu)  01/31/1998
% Modified : Trisha Yannuzzi (trisha@ircs.upenn.edu) 12/28/1999 (in process)
% Modified : Mary Ellen Foster (M.E.Foster@ed.ac.uk) 12/11/2000
% Modified : Ken Forbus                              01/23/2004
% Modified : Eli M. Silk (esilk@pitt.edu)            05/24/2005
% Modified : Niels Taatgen (taatgen@cmu.edu)         10/24/2006
% Modified : David Noelle (dnoelle@ucmerced.edu)     11/19/2014

%% Change "letterpaper" in the following line to "a4paper" if you must.

\documentclass[10pt,letterpaper]{article}

\usepackage{cogsci}
\usepackage{pslatex}
\usepackage{apacite}


\title{Tutorial: Meta-Analytic methods for Cognitive Science}

\author{{\large \bf Sho Tsuji (xyz@abc.def)} \\
  Department of Psychology\\
  Madison, WI 53706 USA
  \And {\large \bf Molly Lewis(xyz@abc.def)} \\
    Department of Psychology\\
    Madison, WI 53706 USA
  \And {\large \bf Mika Braginsky (xyz@abc.def)} \\
      Department of Psychology\\
      Madison, WI 53706 USA
  \AND {\large \bf Christina Bergmann (xyz@abc.def)} \\
        Department of Psychology \\
        Madison, WI 53706 USA
  \And      {\large \bf Page Piccinini (xyz@abc.def)} \\
          Department of Psychology \\
          Madison, WI 53706 USA
  \And        {\large \bf Mike Frank (xyz@abc.def)} \\
            Department of Psychology \\
            Madison, WI 53706 USA
  \AND          {\large \bf Alex Cristia (xyz@abc.def)} \\
              Department of Psychology \\
              Madison, WI 53706 USA}


\begin{document}

\maketitle

\begin{quote}
\small
\textbf{Keywords:}
add your choice of indexing terms or keywords; kindly use a
semicolon; between each term
\end{quote}

\section{General Formatting Instructions}

CogSci abstract

% Tutorials allow participants to gain new insights, knowledge, and skills from a broad range of topics in the field of cognitive science. Tutorials must cover a well-established topic or method, and should be delivered by one or more experts in that area. Tutorials should be presented at a level that will make the material accessible to a postgraduate student with a first degree in a discipline or area of cognitive science. We strongly encourage an interactive delivery format. Tutorials may either be a half-day or full-day in duration. Half-day tutorials are about 3 hours long (not including breaks). Full-day tutorials are about 6 hours long (not including breaks). Proposals should be submitted as two-page summaries that describe the significance of the topic or method, describe the structure and activities to be included in the tutorial, describe the credentials of the tutorial organizer, and include relevant references. (See "Submission Formats," below.) Comments and evaluations from reviewers will be solicited to aid in making decisions about whether or not to accept a given proposal. Organizers of accepted tutorials will be reimbursed for expenses associated with organizing the tutorial, up to a fixed limit ($600 for each half-day tutorial and $1200 for each full-day tutorial that is delivered). If organizers wish to request reimbursement, they need to submit a budget with their proposal. Budgets cannot include travel, hotel, per diems, food/drink (for the session or the organizers), or printed materials. Budgets may include registration costs for individuals that would not otherwise attend the meeting because of demonstrated hardship. Tutorial participants will be charged an extra fee of $30 on top of the regular conference registration. Tutorial organizers will be given access to the email addresses of the preregistered participants so they can contact them in advance about what equipment or supplies to bring, as well as let them know of any preparations that they should make, prior to the tutorial.

\section{Significance}

The past years have presented the cognitive science community with a number of challenges, such as an increased demand for robust research planning [1]. Meta-analyses are a key method to design new studies robustly based on prior results, for instance allowing to base sample size decisions on power analysis. However, conducting a meta-analysis is an effortful study project on its own, and it is not clear that the advantages for an individual researcher justify the resources tied to conducting one.

The MetaLab (\url{http://metalab.stanford.edu}) project proposes three solutions to this problem. First, we provide a package of instruction material, templates, and analysis scripts, streamlining the process of learning about and conducting a meta-analysis [2]. Second, we propose the concept of community-augmented meta-analysis (CAMA) [3], allowing a meta-analysis to be conducted and extended by multiple researchers, both reducing the workload of the individual researcher as well as allowing for dynamic extensions to always include the newest results. Third, for each meta-analysis conducted in the MetaLab framework, we provide free and easy-to-use tools for study planning and power analysis, allowing a whole research community to profit from a meta-analysis once conducted [2]. These extensions to the well-established method of meta-analysis will lower the hurdle of conducting a meta-analysis at the same time as increasing the benefits of a meta-analysis once conducted for a whole research community.

\section{MetaLab Infrastructure}

MetaLab provides spreadsheet templates which include necessary and optional columns and column values.

\section{Structure/Activities}

This one-day tutorial will introduce participants to the method of meta-analysis, providing a hands-on step-to-step guide to use the MetaLab infrastructure for conducting a meta-analysis, working on it collaboratively, and sharing it with the research community.
Participants will connect to the MetaLab infrastructure while following a step-to-step guide to conduct a meta-analysis based on a pre-selected topic. The topics of literature search and study selection, which precede the actual meta-analysis, will be covered briefly, but not included in the hands-on part of the tutorial. Participants will be walked through the steps of a meta-analysis with a theoretical and practical part in each step. Two tutorial organizers will be available for questions and assistance throughout the tutorial.

Coding of variables (2h)
Theory: How to decide on independent and dependent variables to be included; which information is needed
Hands-on: Set-up of a spreadsheet in standardized format, deciding on variables to be included, coding of one pre-selected article (different article for each participant)
Effect size calculation (1h)
Theory: Introduction to different types of effect sizes, their calculation, and how to transform between them
Hands-on: Effect size calculation for paper coded
Meta-analysis (3h)
Theory: Introduction to meta-analytic regression, choice of model, choice of moderator variables, correction for publication bias, and interpretation of analysis output
Hands-on: Putting together the papers coded by each participant and conducting a meta-analysis

\section{Organizer Credentials}

Sho
Molly Lewis
Mika Braginsky
Who else will be there?
%
% References
% [1]
% [2] Bergmann, S., Tsuji, S., Lewis, M., Braginsky, M., Piccinini, P., Cristia, A., & Frank, M. C. (2015). MetaLab: Power Analysis and Experimental Design in Developmental Research Made Easy. Poster presented at the 6th Annual Budapest CEU Conference on Cognitive Development. https://dx.doi.org/10.6084/m9.figshare.2064348.v1
% [3] Tsuji, S., Bergmann, C., & Cristia, A. (2014). Community-Augmented Meta-Analyses Toward Cumulative Data Assessment. Perspectives on Psychological Science, 9(6), 661-665.





\bibliographystyle{apacite}

\setlength{\bibleftmargin}{.125in}
\setlength{\bibindent}{-\bibleftmargin}

\bibliography{metalab_template}


\end{document}
